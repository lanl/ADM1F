%% Generated by Sphinx.
\def\sphinxdocclass{report}
\documentclass[a4paper,10pt,english]{sphinxmanual}
\ifdefined\pdfpxdimen
   \let\sphinxpxdimen\pdfpxdimen\else\newdimen\sphinxpxdimen
\fi \sphinxpxdimen=.75bp\relax

\PassOptionsToPackage{warn}{textcomp}
\usepackage[utf8]{inputenc}
\ifdefined\DeclareUnicodeCharacter
% support both utf8 and utf8x syntaxes
  \ifdefined\DeclareUnicodeCharacterAsOptional
    \def\sphinxDUC#1{\DeclareUnicodeCharacter{"#1}}
  \else
    \let\sphinxDUC\DeclareUnicodeCharacter
  \fi
  \sphinxDUC{00A0}{\nobreakspace}
  \sphinxDUC{2500}{\sphinxunichar{2500}}
  \sphinxDUC{2502}{\sphinxunichar{2502}}
  \sphinxDUC{2514}{\sphinxunichar{2514}}
  \sphinxDUC{251C}{\sphinxunichar{251C}}
  \sphinxDUC{2572}{\textbackslash}
\fi
\usepackage{cmap}
\usepackage[T1]{fontenc}
\usepackage{amsmath,amssymb,amstext}
\usepackage{babel}


\usepackage{amsmath,amsfonts,amssymb,amsthm}

\usepackage{fncychap}
\usepackage{sphinx}

\fvset{fontsize=\small}
\usepackage{geometry}


% Include hyperref last.
\usepackage{hyperref}
% Fix anchor placement for figures with captions.
\usepackage{hypcap}% it must be loaded after hyperref.
% Set up styles of URL: it should be placed after hyperref.
\urlstyle{same}

\addto\captionsenglish{\renewcommand{\contentsname}{User Guide:}}

\usepackage{sphinxmessages}
\setcounter{tocdepth}{1}


% Jupyter Notebook code cell colors
\definecolor{nbsphinxin}{HTML}{307FC1}
\definecolor{nbsphinxout}{HTML}{BF5B3D}
\definecolor{nbsphinx-code-bg}{HTML}{F5F5F5}
\definecolor{nbsphinx-code-border}{HTML}{E0E0E0}
\definecolor{nbsphinx-stderr}{HTML}{FFDDDD}
% ANSI colors for output streams and traceback highlighting
\definecolor{ansi-black}{HTML}{3E424D}
\definecolor{ansi-black-intense}{HTML}{282C36}
\definecolor{ansi-red}{HTML}{E75C58}
\definecolor{ansi-red-intense}{HTML}{B22B31}
\definecolor{ansi-green}{HTML}{00A250}
\definecolor{ansi-green-intense}{HTML}{007427}
\definecolor{ansi-yellow}{HTML}{DDB62B}
\definecolor{ansi-yellow-intense}{HTML}{B27D12}
\definecolor{ansi-blue}{HTML}{208FFB}
\definecolor{ansi-blue-intense}{HTML}{0065CA}
\definecolor{ansi-magenta}{HTML}{D160C4}
\definecolor{ansi-magenta-intense}{HTML}{A03196}
\definecolor{ansi-cyan}{HTML}{60C6C8}
\definecolor{ansi-cyan-intense}{HTML}{258F8F}
\definecolor{ansi-white}{HTML}{C5C1B4}
\definecolor{ansi-white-intense}{HTML}{A1A6B2}
\definecolor{ansi-default-inverse-fg}{HTML}{FFFFFF}
\definecolor{ansi-default-inverse-bg}{HTML}{000000}

% Define an environment for non-plain-text code cell outputs (e.g. images)
\makeatletter
\newenvironment{nbsphinxfancyoutput}{%
    % Avoid fatal error with framed.sty if graphics too long to fit on one page
    \let\sphinxincludegraphics\nbsphinxincludegraphics
    \nbsphinx@image@maxheight\textheight
    \advance\nbsphinx@image@maxheight -2\fboxsep   % default \fboxsep 3pt
    \advance\nbsphinx@image@maxheight -2\fboxrule  % default \fboxrule 0.4pt
    \advance\nbsphinx@image@maxheight -\baselineskip
\def\nbsphinxfcolorbox{\spx@fcolorbox{nbsphinx-code-border}{white}}%
\def\FrameCommand{\nbsphinxfcolorbox\nbsphinxfancyaddprompt\@empty}%
\def\FirstFrameCommand{\nbsphinxfcolorbox\nbsphinxfancyaddprompt\sphinxVerbatim@Continues}%
\def\MidFrameCommand{\nbsphinxfcolorbox\sphinxVerbatim@Continued\sphinxVerbatim@Continues}%
\def\LastFrameCommand{\nbsphinxfcolorbox\sphinxVerbatim@Continued\@empty}%
\MakeFramed{\advance\hsize-\width\@totalleftmargin\z@\linewidth\hsize\@setminipage}%
\lineskip=1ex\lineskiplimit=1ex\raggedright%
}{\par\unskip\@minipagefalse\endMakeFramed}
\makeatother
\newbox\nbsphinxpromptbox
\def\nbsphinxfancyaddprompt{\ifvoid\nbsphinxpromptbox\else
    \kern\fboxrule\kern\fboxsep
    \copy\nbsphinxpromptbox
    \kern-\ht\nbsphinxpromptbox\kern-\dp\nbsphinxpromptbox
    \kern-\fboxsep\kern-\fboxrule\nointerlineskip
    \fi}
\newlength\nbsphinxcodecellspacing
\setlength{\nbsphinxcodecellspacing}{0pt}

% Define support macros for attaching opening and closing lines to notebooks
\newsavebox\nbsphinxbox
\makeatletter
\newcommand{\nbsphinxstartnotebook}[1]{%
    \par
    % measure needed space
    \setbox\nbsphinxbox\vtop{{#1\par}}
    % reserve some space at bottom of page, else start new page
    \needspace{\dimexpr2.5\baselineskip+\ht\nbsphinxbox+\dp\nbsphinxbox}
    % mimick vertical spacing from \section command
      \addpenalty\@secpenalty
      \@tempskipa 3.5ex \@plus 1ex \@minus .2ex\relax
      \addvspace\@tempskipa
      {\Large\@tempskipa\baselineskip
             \advance\@tempskipa-\prevdepth
             \advance\@tempskipa-\ht\nbsphinxbox
             \ifdim\@tempskipa>\z@
               \vskip \@tempskipa
             \fi}
    \unvbox\nbsphinxbox
    % if notebook starts with a \section, prevent it from adding extra space
    \@nobreaktrue\everypar{\@nobreakfalse\everypar{}}%
    % compensate the parskip which will get inserted by next paragraph
    \nobreak\vskip-\parskip
    % do not break here
    \nobreak
}% end of \nbsphinxstartnotebook

\newcommand{\nbsphinxstopnotebook}[1]{%
    \par
    % measure needed space
    \setbox\nbsphinxbox\vbox{{#1\par}}
    \nobreak % it updates page totals
    \dimen@\pagegoal
    \advance\dimen@-\pagetotal \advance\dimen@-\pagedepth
    \advance\dimen@-\ht\nbsphinxbox \advance\dimen@-\dp\nbsphinxbox
    \ifdim\dimen@<\z@
      % little space left
      \unvbox\nbsphinxbox
      \kern-.8\baselineskip
      \nobreak\vskip\z@\@plus1fil
      \penalty100
      \vskip\z@\@plus-1fil
      \kern.8\baselineskip
    \else
      \unvbox\nbsphinxbox
    \fi
}% end of \nbsphinxstopnotebook

% Ensure height of an included graphics fits in nbsphinxfancyoutput frame
\newdimen\nbsphinx@image@maxheight % set in nbsphinxfancyoutput environment
\newcommand*{\nbsphinxincludegraphics}[2][]{%
    \gdef\spx@includegraphics@options{#1}%
    \setbox\spx@image@box\hbox{\includegraphics[#1,draft]{#2}}%
    \in@false
    \ifdim \wd\spx@image@box>\linewidth
      \g@addto@macro\spx@includegraphics@options{,width=\linewidth}%
      \in@true
    \fi
    % no rotation, no need to worry about depth
    \ifdim \ht\spx@image@box>\nbsphinx@image@maxheight
      \g@addto@macro\spx@includegraphics@options{,height=\nbsphinx@image@maxheight}%
      \in@true
    \fi
    \ifin@
      \g@addto@macro\spx@includegraphics@options{,keepaspectratio}%
    \fi
    \setbox\spx@image@box\box\voidb@x % clear memory
    \expandafter\includegraphics\expandafter[\spx@includegraphics@options]{#2}%
}% end of "\MakeFrame"-safe variant of \sphinxincludegraphics
\makeatother

\makeatletter
\renewcommand*\sphinx@verbatim@nolig@list{\do\'\do\`}
\begingroup
\catcode`'=\active
\let\nbsphinx@noligs\@noligs
\g@addto@macro\nbsphinx@noligs{\let'\PYGZsq}
\endgroup
\makeatother
\renewcommand*\sphinxbreaksbeforeactivelist{\do\<\do\"\do\'}
\renewcommand*\sphinxbreaksafteractivelist{\do\.\do\,\do\:\do\;\do\?\do\!\do\/\do\>\do\-}
\makeatletter
\fvset{codes*=\sphinxbreaksattexescapedchars\do\^\^\let\@noligs\nbsphinx@noligs}
\makeatother



\title{ADM1F}
\date{Jun 08, 2021}
\release{0.1}
\author{Satish Karra,\and Kuang Zhu,\and Wenjuan Zhang,\and and Elchin Jafarov}
\newcommand{\sphinxlogo}{\vbox{}}
\renewcommand{\releasename}{ }
\makeindex
\begin{document}

\pagestyle{empty}
\sphinxmaketitle
\pagestyle{plain}
\sphinxtableofcontents
\pagestyle{normal}
\phantomsection\label{\detokenize{index::doc}}


\sphinxAtStartPar
Anaerobic digestion (AD) process converts organic wastes into biogas. Biogas can generate heat and electricity through a cascade of biochemical reactions and has been adapted by various facilities and industries to treat and recover energy from high\sphinxhyphen{}strength liquid or solid waste streams. Anaerobic Digestion Model 1 (ADM1) is a mathematical model that describes the stoichiometry and kinetics of the essential biochemical reactions in AD. This repository includes C++ version of the Matlab/Simulink %
\begin{footnote}[1]\sphinxAtStartFootnote
\sphinxhref{https://pdfs.semanticscholar.org/9f84/13e7bb8ec49b3d0eb321e9d54720f117a527.pdf}{Rosen C., Vrecko D., Gernaey K.V., Pons M.\sphinxhyphen{}N. and Jeppsson U. (2006). Implementing ADM1 for plant\sphinxhyphen{}wide benchmark simulations in Matlab/Simulink. Water Sci. Technol., 54(4), 11\sphinxhyphen{}19.}
%
\end{footnote} version of the ADM1 model and the solid retention time (SRT) version %
\begin{footnote}[2]\sphinxAtStartFootnote
\sphinxtitleref{Zhu et al, in prep. A Novel Core\sphinxhyphen{}shell ADM1 model allows rapid optimization of membrane anaerobic digestion processes}
%
\end{footnote}. The C++ version of the model is computationally more efficient than its Matlab/Simulink predecessor. We called this version of the model Anaerobic Digestion Model 1 Fast (ADM1F).

\noindent\sphinxincludegraphics[width=1216\sphinxpxdimen,height=685\sphinxpxdimen]{{digester_m}.png}


\chapter{Compile ADM1F}
\label{\detokenize{compile:compile-adm1f}}\label{\detokenize{compile::doc}}\begin{enumerate}
\sphinxsetlistlabels{\arabic}{enumi}{enumii}{}{.}%
\item {} 
\sphinxAtStartPar
ADM1F uses external numerical library package PETSc. First download PETSc:

\begin{sphinxVerbatim}[commandchars=\\\{\}]
\PYGZdl{} cd build; git clone \PYGZhy{}b release https://gitlab.com/petsc/petsc.git petsc
\end{sphinxVerbatim}

\begin{sphinxVerbatim}[commandchars=\\\{\}]
\PYGZdl{} cd petsc; git checkout v3.14
\end{sphinxVerbatim}

\item {} 
\sphinxAtStartPar
Set \sphinxstylestrong{PETSC\_DIR} and \sphinxstylestrong{PETSC\_ARCH} in your environmental variables. We suggest to put these lines in your \sphinxtitleref{\textasciitilde{}/.bashrc} or similar files (\sphinxtitleref{\textasciitilde{}/.bash\_profile} on Mac OS X). Once you add it into the bash file, run \sphinxtitleref{source  \textasciitilde{}/.bash\_profile}:

\begin{sphinxVerbatim}[commandchars=\\\{\}]
\PYGZdl{} export PETSC\PYGZus{}DIR=/path\PYGZhy{}to\PYGZhy{}my\PYGZhy{}ADM1F\PYGZhy{}folder/build/petsc
\end{sphinxVerbatim}

\sphinxAtStartPar
and:

\begin{sphinxVerbatim}[commandchars=\\\{\}]
\PYGZdl{} export PETSC\PYGZus{}ARCH=macx\PYGZhy{}debug
\end{sphinxVerbatim}

\sphinxAtStartPar
\sphinxstylestrong{Make sure} that ‘adolc\sphinxhyphen{}utils’ folder is in the ‘build’ folder.

\item {} 
\sphinxAtStartPar
Configure PETSC:

\begin{sphinxVerbatim}[commandchars=\\\{\}]
\PYGZdl{} ./configure \PYGZhy{}\PYGZhy{}download\PYGZhy{}mpich \PYGZhy{}\PYGZhy{}with\PYGZhy{}cc=clang \PYGZhy{}\PYGZhy{}with\PYGZhy{}fc=gfortran \PYGZhy{}\PYGZhy{}with\PYGZhy{}debugging=0 \PYGZhy{}\PYGZhy{}download\PYGZhy{}adolc PETSC\PYGZus{}ARCH=macx\PYGZhy{}debug \PYGZhy{}\PYGZhy{}with\PYGZhy{}cxx\PYGZhy{}dialect=C++11 \PYGZhy{}\PYGZhy{}download\PYGZhy{}colpack
\end{sphinxVerbatim}

\end{enumerate}

\sphinxAtStartPar
\sphinxstylestrong{NOTE}: that these are for Mac OSX. If you are installing on a linux machine, then replace \sphinxstylestrong{clang} with \sphinxstylestrong{gcc}. Also, sometimes turning off \sphinxtitleref{\textendash{}with\sphinxhyphen{}fc=0} could help with compilation. This step will take awhile.
\begin{enumerate}
\sphinxsetlistlabels{\arabic}{enumi}{enumii}{}{.}%
\setcounter{enumi}{3}
\item {} 
\sphinxAtStartPar
If configuration goes well, you can then compile. This step will take awhile too.:

\begin{sphinxVerbatim}[commandchars=\\\{\}]
\PYGZdl{} make PETSC\PYGZus{}DIR=/path\PYGZhy{}to\PYGZhy{}my\PYGZhy{}ADM1F\PYGZhy{}folder/build/petsc PETSC\PYGZus{}ARCH=macx\PYGZhy{}debug all
\end{sphinxVerbatim}

\item {} 
\sphinxAtStartPar
After compilation, PETSc will show you how to test your installation (testing is optional).

\item {} 
\sphinxAtStartPar
Navigate back to the \sphinxtitleref{build} folder (\sphinxtitleref{cd ../}) and compile adm1f:

\begin{sphinxVerbatim}[commandchars=\\\{\}]
\PYGZdl{} make adm1f
\end{sphinxVerbatim}

\sphinxAtStartPar
or:

\begin{sphinxVerbatim}[commandchars=\\\{\}]
\PYGZdl{} make
\end{sphinxVerbatim}

\item {} 
\sphinxAtStartPar
Set \sphinxstylestrong{ADM1F\_EXE} in your environmental variable. Add this line in your \sphinxtitleref{\textasciitilde{}/.bashrc} or similar files (\sphinxtitleref{\textasciitilde{}/.bash\_profile} on Mac OS X).  Once you add it into the bash file, do not forget to \sphinxtitleref{source  \textasciitilde{}/.bash\_profile}:

\begin{sphinxVerbatim}[commandchars=\\\{\}]
\PYGZdl{} export ADM1F\PYGZus{}EXE=path\PYGZhy{}to\PYGZhy{}my\PYGZhy{}ADM1F\PYGZhy{}folder/build/adm1f
\end{sphinxVerbatim}

\item {} 
\sphinxAtStartPar
\sphinxstylestrong{NOTE}: There are two versions of the ADMF1: the original version  (adm1f.cxx), and the modified version of the model (adm1f\_srt.cxx, see {\hyperref[\detokenize{compile:adm1f-srt}]{\sphinxcrossref{\DUrole{std,std-ref}{ADM1F SRT}}}}).

\end{enumerate}


\chapter{Running ADM1F}
\label{\detokenize{compile:running-adm1f}}\begin{enumerate}
\sphinxsetlistlabels{\arabic}{enumi}{enumii}{}{.}%
\item {} 
\sphinxAtStartPar
Make sure that \sphinxstylestrong{ADM1F\_EXE:} is not empty (see step 7 from the previous section).:

\begin{sphinxVerbatim}[commandchars=\\\{\}]
\PYGZdl{} echo \PYGZdl{}ADM1F\PYGZus{}EXE
\end{sphinxVerbatim}

\item {} 
\sphinxAtStartPar
Navigate to the \sphinxtitleref{simulations} folder and run the model:

\begin{sphinxVerbatim}[commandchars=\\\{\}]
\PYGZdl{} \PYGZdl{}ADM1F\PYGZus{}EXE

or using command\PYGZhy{}line options (see 4 and 5):

\PYGZdl{} \PYGZdl{}ADM1F\PYGZus{}EXE \PYGZhy{}ts\PYGZus{}monitor \PYGZhy{}steady
\end{sphinxVerbatim}

\item {} 
\sphinxAtStartPar
Note that adm1f will look for three files \sphinxtitleref{ic.dat}, \sphinxtitleref{params.dat}, and \sphinxtitleref{influent.dat}, which contain the initial conditions (45 values), parameters (100 values), and influent values (28 values), see {\hyperref[\detokenize{inouts:inouts-label}]{\sphinxcrossref{\DUrole{std,std-ref}{Inputs/Outputs}}}}.

\item {} 
\sphinxAtStartPar
The command\sphinxhyphen{}line options are:
\begin{itemize}
\item {} 
\sphinxAtStartPar
\sphinxhyphen{}Cat {[}val{]} \sphinxhyphen{} mass of Cat+ added {[}kmol/m3{]}

\item {} 
\sphinxAtStartPar
\sphinxhyphen{}Vliq {[}val{]} \sphinxhyphen{} volume of liquid {[}m3{]}

\item {} 
\sphinxAtStartPar
\sphinxhyphen{}Vgas {[}val{]} \sphinxhyphen{} volume of liquid {[}m3{]}

\item {} 
\sphinxAtStartPar
\sphinxhyphen{}t\_resx {[}val{]} \sphinxhyphen{}SRT adjustment: t\_resx = SRT\sphinxhyphen{}HRT, {[}d{]} (works only for adm1f\_srt.cxx)

\item {} 
\sphinxAtStartPar
\sphinxhyphen{}params\_file {[}filename{]} \sphinxhyphen{} specify params filename (default is params.dat)

\item {} 
\sphinxAtStartPar
\sphinxhyphen{}ic\_file {[}filename{]} \sphinxhyphen{} specify initial conditions filename (default is ic.dat)

\item {} 
\sphinxAtStartPar
\sphinxhyphen{}influent\_file {[}filename{]} \sphinxhyphen{} specify influent filename (default is influent.dat)

\item {} 
\sphinxAtStartPar
\sphinxhyphen{}ts\_monitor \sphinxhyphen{} shows the timestep and time information on screen

\item {} 
\sphinxAtStartPar
\sphinxhyphen{}steady \sphinxhyphen{} run as steady state else runs as transient

\item {} 
\sphinxAtStartPar
\sphinxhyphen{}debug \sphinxhyphen{} gives out more details on the screen

\end{itemize}

\item {} 
\sphinxAtStartPar
More command\sphinxhyphen{}line options can be found \sphinxhref{https://www.mcs.anl.gov/petsc/petsc-current/docs/manualpages/TS/TSSetFromOptions.html}{here}.

\end{enumerate}


\chapter{ADM1F SRT}
\label{\detokenize{compile:adm1f-srt}}\label{\detokenize{compile:id1}}
\sphinxAtStartPar
The adm1f\_srt.cxx version includes solid retention time (SRT) and other modifications described below. To switch to the SRT version of the model change ‘EXAMPLESC  = adm1f\_srt.cxx’ , ‘OBJECTS\_PF = adm1f\_srt.o’, and ‘adm1f: adm1f\_srt.o’ in the \sphinxtitleref{build/makefile}. Then recompile the model (Compile ADM1F, step 6).
\begin{itemize}
\item {} 
\sphinxAtStartPar
Includes a term (T\_resx) in the mass balance to separate the solids retention time from hydraulic retention time.

\item {} 
\sphinxAtStartPar
Uses the empirical Hill function that describes the inhibition of acetogenesis and hydrogenotrophic methanogenesis by acetic acid with the noncompetitive inhibition model %
\begin{footnote}[1]\sphinxAtStartFootnote
Love, N. G., R. J. Smith, K. R. Gilmore, and C. W. Randall. 1999. Oxime inhibition of nitrification during treatment of an ammonia\sphinxhyphen{}containing industrial waste. Water Environment Research 71:418\textendash{}26.
%
\end{footnote} %
\begin{footnote}[2]\sphinxAtStartFootnote
Oslislo, A., and Z. Lewandowski. 1985. Inhibition of nitrification in the packed bed reactors by selected organic compounds. Water Research 19:423\textendash{}26.
%
\end{footnote}.

\item {} 
\sphinxAtStartPar
Describes the inhibition of acetic acid on acetoclastic methanogenesis with the Haldane equation %
\begin{footnote}[3]\sphinxAtStartFootnote
Haldane, J. B. S. 1930. Enzymes. London: Longmans.
%
\end{footnote} %
\begin{footnote}[4]\sphinxAtStartFootnote
Andrews, J. F. 1968. A mathematical model for the continuous culture of microorganisms utilizing inhibitory substrates. Biotechnology and Bioengineering 10:707\textendash{}23.
%
\end{footnote}.

\item {} 
\sphinxAtStartPar
Includes a adsorption\sphinxhyphen{}inhibition term describing the long\sphinxhyphen{}chain fatty acid %
\begin{footnote}[6]\sphinxAtStartFootnote
Palatsi, J., Illa, J., Prenafeta\sphinxhyphen{}Boldú, F.X., Laureni, M., Fernandez, B., Angelidaki, I., Flotats. X. 2010. Long\sphinxhyphen{}chain fatty acids inhibition and adaptation process in anaerobic thermophilic digestion: Batch tests, microbial community structure and mathematical modelling. Bioresource Technology. 101, 7, 2243\sphinxhyphen{}2251.
%
\end{footnote} (LCFA) inhibition of LCFA degradation and methanogenesis.

\item {} 
\sphinxAtStartPar
Includes Arrhenius equations describing the effect of temperature on bioreaction kinetics %
\begin{footnote}[5]\sphinxAtStartFootnote
Novak, J. T. 1974. Temperature\sphinxhyphen{}substrate interactions in biological treatment. Journal, Water Pollution Control Federation 46:1984\textendash{}94.
%
\end{footnote}.

\item {} 
\sphinxAtStartPar
Includes a cation term to simulate the addition of NaOH for pH adjustment.

\end{itemize}
\subsubsection*{References}


\chapter{Inputs/Outputs}
\label{\detokenize{inouts:inputs-outputs}}\label{\detokenize{inouts:inouts-label}}\label{\detokenize{inouts::doc}}

\section{Initial Conditions {[}ic.dat{]}}
\label{\detokenize{inouts:initial-conditions-ic-dat}}

\begin{savenotes}\sphinxatlongtablestart\begin{longtable}[c]{|\X{10}{120}|\X{20}{120}|\X{30}{120}|\X{60}{120}|}
\hline
\sphinxstyletheadfamily 
\sphinxAtStartPar
Index
&\sphinxstyletheadfamily 
\sphinxAtStartPar
Notation
&\sphinxstyletheadfamily 
\sphinxAtStartPar
Unit
&\sphinxstyletheadfamily 
\sphinxAtStartPar
Description
\\
\hline
\endfirsthead

\multicolumn{4}{c}%
{\makebox[0pt]{\sphinxtablecontinued{\tablename\ \thetable{} \textendash{} continued from previous page}}}\\
\hline
\sphinxstyletheadfamily 
\sphinxAtStartPar
Index
&\sphinxstyletheadfamily 
\sphinxAtStartPar
Notation
&\sphinxstyletheadfamily 
\sphinxAtStartPar
Unit
&\sphinxstyletheadfamily 
\sphinxAtStartPar
Description
\\
\hline
\endhead

\hline
\multicolumn{4}{r}{\makebox[0pt][r]{\sphinxtablecontinued{continues on next page}}}\\
\endfoot

\endlastfoot

\sphinxAtStartPar
1
&
\sphinxAtStartPar
S\_su
&
\sphinxAtStartPar
kgCOD/m3
&
\sphinxAtStartPar
soluble  monosaccharides
\\
\hline
\sphinxAtStartPar
2
&
\sphinxAtStartPar
S\_aa
&
\sphinxAtStartPar
kgCOD/m3
&
\sphinxAtStartPar
soluble  amino acids
\\
\hline
\sphinxAtStartPar
3
&
\sphinxAtStartPar
S\_fa
&
\sphinxAtStartPar
kgCOD/m3
&
\sphinxAtStartPar
soluble  total LCFA
\\
\hline
\sphinxAtStartPar
4
&
\sphinxAtStartPar
S\_va
&
\sphinxAtStartPar
kgCOD/m3
&
\sphinxAtStartPar
soluble  total valerate
\\
\hline
\sphinxAtStartPar
5
&
\sphinxAtStartPar
S\_bu
&
\sphinxAtStartPar
kgCOD/m3
&
\sphinxAtStartPar
soluble  total butyrate
\\
\hline
\sphinxAtStartPar
6
&
\sphinxAtStartPar
S\_pro
&
\sphinxAtStartPar
kgCOD/m3
&
\sphinxAtStartPar
soluble  total propionate
\\
\hline
\sphinxAtStartPar
7
&
\sphinxAtStartPar
S\_ac
&
\sphinxAtStartPar
kgCOD/m3
&
\sphinxAtStartPar
soluble  acetate
\\
\hline
\sphinxAtStartPar
8
&
\sphinxAtStartPar
S\_h2
&
\sphinxAtStartPar
kgCOD/m3
&
\sphinxAtStartPar
hydrogen gas
\\
\hline
\sphinxAtStartPar
9
&
\sphinxAtStartPar
S\_ch4
&
\sphinxAtStartPar
kgCOD/m3
&
\sphinxAtStartPar
methane gas
\\
\hline
\sphinxAtStartPar
10
&
\sphinxAtStartPar
S\_IC
&
\sphinxAtStartPar
kmoleC/m3
&
\sphinxAtStartPar
soluble inorganic carbon
\\
\hline
\sphinxAtStartPar
11
&
\sphinxAtStartPar
S\_IN
&
\sphinxAtStartPar
kmoleC/m3
&
\sphinxAtStartPar
soluble inorganic nitrogen
\\
\hline
\sphinxAtStartPar
12
&
\sphinxAtStartPar
S\_I
&
\sphinxAtStartPar
kgCOD/m3
&
\sphinxAtStartPar
soluble inert materials
\\
\hline
\sphinxAtStartPar
13
&
\sphinxAtStartPar
X\_c\_biom
&
\sphinxAtStartPar
kgCOD/m3
&
\sphinxAtStartPar
particulate  of composites
\\
\hline
\sphinxAtStartPar
14
&
\sphinxAtStartPar
X\_ch\_biom
&
\sphinxAtStartPar
kgCOD/m3
&
\sphinxAtStartPar
particulate  of carbohydrate
\\
\hline
\sphinxAtStartPar
15
&
\sphinxAtStartPar
X\_pr\_biom
&
\sphinxAtStartPar
kgCOD/m3
&
\sphinxAtStartPar
particulate  of proteins
\\
\hline
\sphinxAtStartPar
16
&
\sphinxAtStartPar
X\_li\_biom
&
\sphinxAtStartPar
kgCOD/m3
&
\sphinxAtStartPar
particulate  of lipids
\\
\hline
\sphinxAtStartPar
17
&
\sphinxAtStartPar
X\_su
&
\sphinxAtStartPar
kgCOD/m3
&
\sphinxAtStartPar
monosaccharides degraders (microorganisms)
\\
\hline
\sphinxAtStartPar
18
&
\sphinxAtStartPar
X\_aa
&
\sphinxAtStartPar
kgCOD/m3
&
\sphinxAtStartPar
amino acids degraders (microorganisms)
\\
\hline
\sphinxAtStartPar
19
&
\sphinxAtStartPar
X\_fa
&
\sphinxAtStartPar
kgCOD/m3
&
\sphinxAtStartPar
LCFA degraders (microorganisms)
\\
\hline
\sphinxAtStartPar
20
&
\sphinxAtStartPar
X\_c4
&
\sphinxAtStartPar
kgCOD/m3
&
\sphinxAtStartPar
valerate and butyrate degraders (microorganisms)
\\
\hline
\sphinxAtStartPar
21
&
\sphinxAtStartPar
X\_pro
&
\sphinxAtStartPar
kgCOD/m3
&
\sphinxAtStartPar
propionate degraders (microorganisms)
\\
\hline
\sphinxAtStartPar
22
&
\sphinxAtStartPar
X\_ac
&
\sphinxAtStartPar
kgCOD/m3
&
\sphinxAtStartPar
acetate degraders (microorganisms)
\\
\hline
\sphinxAtStartPar
23
&
\sphinxAtStartPar
X\_h2
&
\sphinxAtStartPar
kgCOD/m3
&
\sphinxAtStartPar
hydrogen degraders (microorganisms)
\\
\hline
\sphinxAtStartPar
24
&
\sphinxAtStartPar
X\_I
&
\sphinxAtStartPar
kgCOD/m3
&
\sphinxAtStartPar
particulate  of inerts
\\
\hline
\sphinxAtStartPar
25
&
\sphinxAtStartPar
S\_cation
&
\sphinxAtStartPar
kmole/m3
&
\sphinxAtStartPar
cations (strong base)
\\
\hline
\sphinxAtStartPar
26
&
\sphinxAtStartPar
S\_anion
&
\sphinxAtStartPar
kmole/m3
&
\sphinxAtStartPar
anions (strong acid)
\\
\hline
\sphinxAtStartPar
27
&
\sphinxAtStartPar
S\_hva
&
\sphinxAtStartPar
kgCOD/m3
&
\sphinxAtStartPar
soluble  valerate acid
\\
\hline
\sphinxAtStartPar
28
&
\sphinxAtStartPar
S\_hbu
&
\sphinxAtStartPar
kgCOD/m3
&
\sphinxAtStartPar
soluble  butyratic acid
\\
\hline
\sphinxAtStartPar
29
&
\sphinxAtStartPar
S\_hpro
&
\sphinxAtStartPar
kgCOD/m3
&
\sphinxAtStartPar
soluble  propionatic acid
\\
\hline
\sphinxAtStartPar
30
&
\sphinxAtStartPar
S\_hac
&
\sphinxAtStartPar
kgCOD/m3
&
\sphinxAtStartPar
soluble  acetatic acid
\\
\hline
\sphinxAtStartPar
31
&
\sphinxAtStartPar
S\_hco3
&
\sphinxAtStartPar
kmole/m3
&
\sphinxAtStartPar
soluble  bicarbonate
\\
\hline
\sphinxAtStartPar
32
&
\sphinxAtStartPar
S\_nh3
&
\sphinxAtStartPar
kmole/m3
&
\sphinxAtStartPar
soluble  ammonia
\\
\hline
\sphinxAtStartPar
33
&
\sphinxAtStartPar
S\_gas\_h2
&
\sphinxAtStartPar
kgCOD/m3
&
\sphinxAtStartPar
soluble  hydrogen gas
\\
\hline
\sphinxAtStartPar
34
&
\sphinxAtStartPar
S\_gas\_ch4
&
\sphinxAtStartPar
kgCOD/m3
&
\sphinxAtStartPar
soluble  methane gas
\\
\hline
\sphinxAtStartPar
35
&
\sphinxAtStartPar
S\_gas\_co2
&
\sphinxAtStartPar
kmole/m3
&
\sphinxAtStartPar
soluble  carbon dioxide gas
\\
\hline
\sphinxAtStartPar
36
&
\sphinxAtStartPar
Q
&
\sphinxAtStartPar
m3/d
&
\sphinxAtStartPar
flow rate
\\
\hline
\sphinxAtStartPar
37
&
\sphinxAtStartPar
Temp
&
\sphinxAtStartPar
°C
&
\sphinxAtStartPar
temperature
\\
\hline
\sphinxAtStartPar
38
&
\sphinxAtStartPar
S\_D1\_D
&
\sphinxAtStartPar
unitless
&
\sphinxAtStartPar
Dummy
\\
\hline
\sphinxAtStartPar
39
&
\sphinxAtStartPar
S\_D2\_D
&
\sphinxAtStartPar
unitless
&
\sphinxAtStartPar
Dummy
\\
\hline
\sphinxAtStartPar
40
&
\sphinxAtStartPar
S\_D3\_D
&
\sphinxAtStartPar
unitless
&
\sphinxAtStartPar
Dummy
\\
\hline
\sphinxAtStartPar
41
&
\sphinxAtStartPar
X\_D4\_D
&
\sphinxAtStartPar
unitless
&
\sphinxAtStartPar
Dummy
\\
\hline
\sphinxAtStartPar
42
&
\sphinxAtStartPar
X\_D5\_D
&
\sphinxAtStartPar
unitless
&
\sphinxAtStartPar
Dummy
\\
\hline
\sphinxAtStartPar
43
&
\sphinxAtStartPar
S\_H\_ion
&
\sphinxAtStartPar
kmoleH+/m3
&
\sphinxAtStartPar
soluble  hydrogen ion
\\
\hline
\sphinxAtStartPar
44
&
\sphinxAtStartPar
S\_co2
&
\sphinxAtStartPar
kmoleC/m3
&
\sphinxAtStartPar
soluble  carbon dioxide
\\
\hline
\sphinxAtStartPar
45
&
\sphinxAtStartPar
S\_nh4
&
\sphinxAtStartPar
kmoleN/m3
&
\sphinxAtStartPar
soluble  ammonium
\\
\hline
\end{longtable}\sphinxatlongtableend\end{savenotes}


\section{Influent Data {[}influent.dat{]}}
\label{\detokenize{inouts:influent-data-influent-dat}}

\begin{savenotes}\sphinxattablestart
\centering
\begin{tabular}[t]{|\X{10}{120}|\X{20}{120}|\X{30}{120}|\X{60}{120}|}
\hline
\sphinxstyletheadfamily 
\sphinxAtStartPar
Index
&\sphinxstyletheadfamily 
\sphinxAtStartPar
Notation
&\sphinxstyletheadfamily 
\sphinxAtStartPar
Unit
&\sphinxstyletheadfamily 
\sphinxAtStartPar
Description
\\
\hline
\sphinxAtStartPar
1
&
\sphinxAtStartPar
S\_su\_in
&
\sphinxAtStartPar
kgCOD/m3
&
\sphinxAtStartPar
soluble  input monosaccharides
\\
\hline
\sphinxAtStartPar
2
&
\sphinxAtStartPar
S\_aa\_in
&
\sphinxAtStartPar
kgCOD/m3
&
\sphinxAtStartPar
soluble  input amino acids
\\
\hline
\sphinxAtStartPar
3
&
\sphinxAtStartPar
S\_fa\_in
&
\sphinxAtStartPar
kgCOD/m3
&
\sphinxAtStartPar
soluble  input total LCFA
\\
\hline
\sphinxAtStartPar
4
&
\sphinxAtStartPar
S\_va\_in
&
\sphinxAtStartPar
kgCOD/m3
&
\sphinxAtStartPar
soluble  input total valerate
\\
\hline
\sphinxAtStartPar
5
&
\sphinxAtStartPar
S\_bu\_in
&
\sphinxAtStartPar
kgCOD/m3
&
\sphinxAtStartPar
soluble  input total butyrate
\\
\hline
\sphinxAtStartPar
6
&
\sphinxAtStartPar
S\_pro\_in
&
\sphinxAtStartPar
kgCOD/m3
&
\sphinxAtStartPar
soluble  input total propionate
\\
\hline
\sphinxAtStartPar
7
&
\sphinxAtStartPar
S\_ac\_in
&
\sphinxAtStartPar
kgCOD/m3
&
\sphinxAtStartPar
soluble input acetate
\\
\hline
\sphinxAtStartPar
8
&
\sphinxAtStartPar
S\_h2\_in
&
\sphinxAtStartPar
kgCOD/m3
&
\sphinxAtStartPar
hydrogen gas
\\
\hline
\sphinxAtStartPar
9
&
\sphinxAtStartPar
S\_ch4\_in
&
\sphinxAtStartPar
kgCOD/m3
&
\sphinxAtStartPar
methane gas
\\
\hline
\sphinxAtStartPar
10
&
\sphinxAtStartPar
S\_IC\_in
&
\sphinxAtStartPar
kmoleC/m3
&
\sphinxAtStartPar
soluble input inorganic carbon
\\
\hline
\sphinxAtStartPar
11
&
\sphinxAtStartPar
S\_IN\_in
&
\sphinxAtStartPar
kmoleC/m3
&
\sphinxAtStartPar
soluble input inorganic nitrogen
\\
\hline
\sphinxAtStartPar
12
&
\sphinxAtStartPar
S\_I \_in
&
\sphinxAtStartPar
kgCOD/m3
&
\sphinxAtStartPar
soluble input inert materials
\\
\hline
\sphinxAtStartPar
13
&
\sphinxAtStartPar
X\_c\_biom\_in
&
\sphinxAtStartPar
kgCOD/m3
&
\sphinxAtStartPar
particulate input of composites
\\
\hline
\sphinxAtStartPar
14
&
\sphinxAtStartPar
X\_ch\_biom\_in
&
\sphinxAtStartPar
kgCOD/m3
&
\sphinxAtStartPar
particulate input of carbohydrate
\\
\hline
\sphinxAtStartPar
15
&
\sphinxAtStartPar
X\_pr\_biom\_in
&
\sphinxAtStartPar
kgCOD/m3
&
\sphinxAtStartPar
particulate input of proteins
\\
\hline
\sphinxAtStartPar
16
&
\sphinxAtStartPar
X\_li\_biom\_in
&
\sphinxAtStartPar
kgCOD/m3
&
\sphinxAtStartPar
particulate input of lipids
\\
\hline
\sphinxAtStartPar
17
&
\sphinxAtStartPar
X\_su\_in
&
\sphinxAtStartPar
kgCOD/m3
&
\sphinxAtStartPar
monosaccharides degraders (microorganisms)
\\
\hline
\sphinxAtStartPar
18
&
\sphinxAtStartPar
X\_aa\_in
&
\sphinxAtStartPar
kgCOD/m3
&
\sphinxAtStartPar
amino acids degraders (microorganisms)
\\
\hline
\sphinxAtStartPar
19
&
\sphinxAtStartPar
X\_fa\_in
&
\sphinxAtStartPar
kgCOD/m3
&
\sphinxAtStartPar
LCFA degraders (microorganisms)
\\
\hline
\sphinxAtStartPar
20
&
\sphinxAtStartPar
X\_c4\_in
&
\sphinxAtStartPar
kgCOD/m3
&
\sphinxAtStartPar
valerate and butyrate degraders (microorganisms)
\\
\hline
\sphinxAtStartPar
21
&
\sphinxAtStartPar
X\_pro\_in
&
\sphinxAtStartPar
kgCOD/m3
&
\sphinxAtStartPar
propionate degraders (microorganisms)
\\
\hline
\sphinxAtStartPar
22
&
\sphinxAtStartPar
X\_ac\_in
&
\sphinxAtStartPar
kgCOD/m3
&
\sphinxAtStartPar
acetate degraders (microorganisms)
\\
\hline
\sphinxAtStartPar
23
&
\sphinxAtStartPar
X\_h2\_in
&
\sphinxAtStartPar
kgCOD/m3
&
\sphinxAtStartPar
hydrogen degraders (microorganisms)
\\
\hline
\sphinxAtStartPar
24
&
\sphinxAtStartPar
X\_I\_in
&
\sphinxAtStartPar
kgCOD/m3
&
\sphinxAtStartPar
particulate input of inerts
\\
\hline
\sphinxAtStartPar
25
&
\sphinxAtStartPar
S\_cation\_in
&
\sphinxAtStartPar
kmole/m3
&
\sphinxAtStartPar
input cations
\\
\hline
\sphinxAtStartPar
26
&
\sphinxAtStartPar
S\_anion\_in
&
\sphinxAtStartPar
kmole/m3
&
\sphinxAtStartPar
input anions
\\
\hline
\sphinxAtStartPar
27
&
\sphinxAtStartPar
Q
&
\sphinxAtStartPar
m3/d
&
\sphinxAtStartPar
flow rate
\\
\hline
\sphinxAtStartPar
28
&
\sphinxAtStartPar
Temp
&
\sphinxAtStartPar
°C
&
\sphinxAtStartPar
temperature
\\
\hline
\end{tabular}
\par
\sphinxattableend\end{savenotes}


\section{Parameters {[}params.dat{]}}
\label{\detokenize{inouts:parameters-params-dat}}

\begin{savenotes}\sphinxatlongtablestart\begin{longtable}[c]{|\X{10}{120}|\X{20}{120}|\X{30}{120}|\X{60}{120}|}
\hline
\sphinxstyletheadfamily 
\sphinxAtStartPar
Index
&\sphinxstyletheadfamily 
\sphinxAtStartPar
Notation
&\sphinxstyletheadfamily 
\sphinxAtStartPar
Unit
&\sphinxstyletheadfamily 
\sphinxAtStartPar
Description
\\
\hline
\endfirsthead

\multicolumn{4}{c}%
{\makebox[0pt]{\sphinxtablecontinued{\tablename\ \thetable{} \textendash{} continued from previous page}}}\\
\hline
\sphinxstyletheadfamily 
\sphinxAtStartPar
Index
&\sphinxstyletheadfamily 
\sphinxAtStartPar
Notation
&\sphinxstyletheadfamily 
\sphinxAtStartPar
Unit
&\sphinxstyletheadfamily 
\sphinxAtStartPar
Description
\\
\hline
\endhead

\hline
\multicolumn{4}{r}{\makebox[0pt][r]{\sphinxtablecontinued{continues on next page}}}\\
\endfoot

\endlastfoot

\sphinxAtStartPar
1
&
\sphinxAtStartPar
f\_sI\_xc
&
\sphinxAtStartPar
kgCOD/kg COD
&
\sphinxAtStartPar
fraction of composites (substrate) disintegrate to soluble inerts (product)
\\
\hline
\sphinxAtStartPar
2
&
\sphinxAtStartPar
f\_xI\_xc
&
\sphinxAtStartPar
kgCOD/kg COD
&
\sphinxAtStartPar
fraction of composites (substrate) disintegrate to particulate inerts (product)
\\
\hline
\sphinxAtStartPar
3
&
\sphinxAtStartPar
f\_ch\_xc
&
\sphinxAtStartPar
kgCOD/kg COD
&
\sphinxAtStartPar
fraction of composites (substrate) disintegrate to carbohydrates (product)
\\
\hline
\sphinxAtStartPar
4
&
\sphinxAtStartPar
f\_pr\_xc
&
\sphinxAtStartPar
kgCOD/kg COD
&
\sphinxAtStartPar
fraction of composites (substrate) disintegrate to proteins (product)
\\
\hline
\sphinxAtStartPar
5
&
\sphinxAtStartPar
f\_li\_xc
&
\sphinxAtStartPar
kgCOD/kg COD
&
\sphinxAtStartPar
fraction of composites (substrate) disintegrate to lipids (product)
\\
\hline
\sphinxAtStartPar
6
&
\sphinxAtStartPar
N\_xc
&
\sphinxAtStartPar
kmole N/(kg COD)
&
\sphinxAtStartPar
nitrogen content of  composites
\\
\hline
\sphinxAtStartPar
7
&
\sphinxAtStartPar
N\_I
&
\sphinxAtStartPar
kmole N/(kg COD)
&
\sphinxAtStartPar
nitrogen content of  inerts
\\
\hline
\sphinxAtStartPar
8
&
\sphinxAtStartPar
N\_aa
&
\sphinxAtStartPar
kmole N/(kg COD)
&
\sphinxAtStartPar
nitrogen content of  amino acids
\\
\hline
\sphinxAtStartPar
9
&
\sphinxAtStartPar
C\_xc
&
\sphinxAtStartPar
kmole C/(kg COD)
&
\sphinxAtStartPar
carbon content of  composites
\\
\hline
\sphinxAtStartPar
10
&
\sphinxAtStartPar
C\_sI
&
\sphinxAtStartPar
kmole C/(kg COD)
&
\sphinxAtStartPar
carbon content of  soluble inerts
\\
\hline
\sphinxAtStartPar
11
&
\sphinxAtStartPar
C\_ch
&
\sphinxAtStartPar
kmole C/(kg COD)
&
\sphinxAtStartPar
carbon content of  carbohydrates
\\
\hline
\sphinxAtStartPar
12
&
\sphinxAtStartPar
C\_pr
&
\sphinxAtStartPar
kmole C/(kg COD)
&
\sphinxAtStartPar
carbon content of  proteins
\\
\hline
\sphinxAtStartPar
13
&
\sphinxAtStartPar
C\_li
&
\sphinxAtStartPar
kmole C/(kg COD)
&
\sphinxAtStartPar
carbon content of  lipids
\\
\hline
\sphinxAtStartPar
14
&
\sphinxAtStartPar
C\_xI
&
\sphinxAtStartPar
kmole C/(kg COD)
&
\sphinxAtStartPar
carbon content of  particulate inerts
\\
\hline
\sphinxAtStartPar
15
&
\sphinxAtStartPar
C\_su
&
\sphinxAtStartPar
kmole C/(kg COD)
&
\sphinxAtStartPar
carbon content of  monosaccharides
\\
\hline
\sphinxAtStartPar
16
&
\sphinxAtStartPar
C\_aa
&
\sphinxAtStartPar
kmole C/(kg COD)
&
\sphinxAtStartPar
carbon content of  amino acids
\\
\hline
\sphinxAtStartPar
17
&
\sphinxAtStartPar
f\_fa\_li
&
\sphinxAtStartPar
kgCOD/kg COD
&
\sphinxAtStartPar
fraction of lipids (substrate) degrade to LCFA (product)
\\
\hline
\sphinxAtStartPar
18
&
\sphinxAtStartPar
C\_fa
&
\sphinxAtStartPar
kmole C/(kg COD)
&
\sphinxAtStartPar
carbon content of  total LCFA
\\
\hline
\sphinxAtStartPar
19
&
\sphinxAtStartPar
f\_h2\_su
&
\sphinxAtStartPar
kgCOD/kg COD
&
\sphinxAtStartPar
fraction of monosaccharides (substrate) degrade to hydrogen (product)
\\
\hline
\sphinxAtStartPar
20
&
\sphinxAtStartPar
f\_bu\_su
&
\sphinxAtStartPar
kgCOD/kg COD
&
\sphinxAtStartPar
fraction of monosaccharides (substrate) degrade to butyrate (product)
\\
\hline
\sphinxAtStartPar
21
&
\sphinxAtStartPar
f\_pro\_su
&
\sphinxAtStartPar
kgCOD/kg COD
&
\sphinxAtStartPar
fraction of monosaccharides (substrate) degrade to propionate (product)
\\
\hline
\sphinxAtStartPar
22
&
\sphinxAtStartPar
f\_ac\_su
&
\sphinxAtStartPar
kgCOD/kg COD
&
\sphinxAtStartPar
fraction of monosaccharides (substrate) degrade to acetate (product)
\\
\hline
\sphinxAtStartPar
23
&
\sphinxAtStartPar
N\_bac
&
\sphinxAtStartPar
kmole N/(kg COD)
&
\sphinxAtStartPar
nitrogen content of synthesized into bacteria
\\
\hline
\sphinxAtStartPar
24
&
\sphinxAtStartPar
C\_bu
&
\sphinxAtStartPar
kmole C/(kg COD)
&
\sphinxAtStartPar
carbon content of  total butyrate
\\
\hline
\sphinxAtStartPar
25
&
\sphinxAtStartPar
C\_pro
&
\sphinxAtStartPar
kmole C/(kg COD)
&
\sphinxAtStartPar
carbon content of  total propionate
\\
\hline
\sphinxAtStartPar
26
&
\sphinxAtStartPar
C\_ac
&
\sphinxAtStartPar
kmole C/(kg COD)
&
\sphinxAtStartPar
carbon content of  total acetate
\\
\hline
\sphinxAtStartPar
27
&
\sphinxAtStartPar
C\_bac
&
\sphinxAtStartPar
kmole C/(kg COD)
&
\sphinxAtStartPar
carbon content of  synthesized into bacteria
\\
\hline
\sphinxAtStartPar
28
&
\sphinxAtStartPar
Y\_su
&
\sphinxAtStartPar
kgCOD\_X/kg COD\_S
&
\sphinxAtStartPar
yield of biomass on monosaccharides (substrate)
\\
\hline
\sphinxAtStartPar
29
&
\sphinxAtStartPar
f\_h2\_aa
&
\sphinxAtStartPar
kgCOD/kg COD
&
\sphinxAtStartPar
fraction of amino acids (substrate) degrade to hydrogen (product)
\\
\hline
\sphinxAtStartPar
30
&
\sphinxAtStartPar
f\_va\_aa
&
\sphinxAtStartPar
kgCOD/kg COD
&
\sphinxAtStartPar
fraction of amino acids (substrate) degrade to valerate (product)
\\
\hline
\sphinxAtStartPar
31
&
\sphinxAtStartPar
f\_bu\_aa
&
\sphinxAtStartPar
kgCOD/kg COD
&
\sphinxAtStartPar
fraction of amino acids (substrate) degrade to butyrate (product)
\\
\hline
\sphinxAtStartPar
32
&
\sphinxAtStartPar
f\_pro\_aa
&
\sphinxAtStartPar
kgCOD/kg COD
&
\sphinxAtStartPar
fraction of amino acids (substrate) degrade to propionate (product)
\\
\hline
\sphinxAtStartPar
33
&
\sphinxAtStartPar
f\_ac\_aa
&
\sphinxAtStartPar
kgCOD/kg COD
&
\sphinxAtStartPar
fraction of amino acids (substrate) degrade to acetate (product)
\\
\hline
\sphinxAtStartPar
34
&
\sphinxAtStartPar
C\_va
&
\sphinxAtStartPar
kmole C/(kg COD)
&
\sphinxAtStartPar
carbon content of  total valerate
\\
\hline
\sphinxAtStartPar
35
&
\sphinxAtStartPar
Y\_aa
&
\sphinxAtStartPar
kgCOD\_X/kg COD\_S
&
\sphinxAtStartPar
yield of biomass on amino acids (substrate)
\\
\hline
\sphinxAtStartPar
36
&
\sphinxAtStartPar
Y\_fa
&
\sphinxAtStartPar
kgCOD\_X/kg COD\_S
&
\sphinxAtStartPar
yield of biomass on total LCFA (substrate)
\\
\hline
\sphinxAtStartPar
37
&
\sphinxAtStartPar
Y\_c4
&
\sphinxAtStartPar
kgCOD\_X/kg COD\_S
&
\sphinxAtStartPar
yield of biomass on butyrate (substrate)
\\
\hline
\sphinxAtStartPar
38
&
\sphinxAtStartPar
Y\_pro
&
\sphinxAtStartPar
kgCOD\_X/kg COD\_S
&
\sphinxAtStartPar
yield of biomass on total propionate (substrate)
\\
\hline
\sphinxAtStartPar
39
&
\sphinxAtStartPar
C\_ch4
&
\sphinxAtStartPar
kmole C/(kg COD)
&
\sphinxAtStartPar
carbon content of methane
\\
\hline
\sphinxAtStartPar
40
&
\sphinxAtStartPar
Y\_ac
&
\sphinxAtStartPar
kgCOD\_X/kg COD\_S
&
\sphinxAtStartPar
yield of biomass on total acetate (substrate)
\\
\hline
\sphinxAtStartPar
41
&
\sphinxAtStartPar
Y\_h2
&
\sphinxAtStartPar
kgCOD\_X/kg COD\_S
&
\sphinxAtStartPar
yield of biomass on hydrogen (substrate)
\\
\hline
\sphinxAtStartPar
42
&
\sphinxAtStartPar
k\_dis
&
\sphinxAtStartPar
1/d
&
\sphinxAtStartPar
disintegration rate
\\
\hline
\sphinxAtStartPar
43
&
\sphinxAtStartPar
k\_hyd\_ch
&
\sphinxAtStartPar
1/d
&
\sphinxAtStartPar
hydrolysis rate of carbohydrates (carbs to simple sugars)
\\
\hline
\sphinxAtStartPar
44
&
\sphinxAtStartPar
k\_hyd\_pr
&
\sphinxAtStartPar
1/d
&
\sphinxAtStartPar
hydrolysis rate of proteins
\\
\hline
\sphinxAtStartPar
45
&
\sphinxAtStartPar
k\_hyd\_li
&
\sphinxAtStartPar
1/d
&
\sphinxAtStartPar
hydrolysis rate of lipids
\\
\hline
\sphinxAtStartPar
46
&
\sphinxAtStartPar
K\_S\_IN
&
\sphinxAtStartPar
kgCOD\_S/m3
&
\sphinxAtStartPar
half saturation value of inorganic nitrogen
\\
\hline
\sphinxAtStartPar
47
&
\sphinxAtStartPar
k\_m\_su
&
\sphinxAtStartPar
kgCOD\_S/kgCOD\_X/d
&
\sphinxAtStartPar
Monod maximum specific uptake rate for monosaccharides
\\
\hline
\sphinxAtStartPar
48
&
\sphinxAtStartPar
K\_S\_su
&
\sphinxAtStartPar
kgCOD\_S/m3
&
\sphinxAtStartPar
half saturation value of monosaccharides
\\
\hline
\sphinxAtStartPar
49
&
\sphinxAtStartPar
pH\_UL\_acidacet
&
\sphinxAtStartPar
unitless
&
\sphinxAtStartPar
upper pH limit of acidic acetate
\\
\hline
\sphinxAtStartPar
50
&
\sphinxAtStartPar
pH\_LL\_acidacet
&
\sphinxAtStartPar
unitless
&
\sphinxAtStartPar
lower pH limit of acidic acetate
\\
\hline
\sphinxAtStartPar
51
&
\sphinxAtStartPar
k\_m\_aa
&
\sphinxAtStartPar
kgCOD\_S/kgCOD\_X/d
&
\sphinxAtStartPar
Monod maximum specific uptake rate for amino acids
\\
\hline
\sphinxAtStartPar
52
&
\sphinxAtStartPar
K\_S\_aa
&
\sphinxAtStartPar
kgCOD\_S/m3
&
\sphinxAtStartPar
half saturation value of amino acids
\\
\hline
\sphinxAtStartPar
53
&
\sphinxAtStartPar
k\_m\_fa
&
\sphinxAtStartPar
kgCOD\_S/kgCOD\_X/d
&
\sphinxAtStartPar
Monod maximum specific uptake rate for total LCFA
\\
\hline
\sphinxAtStartPar
54
&
\sphinxAtStartPar
K\_S\_fa
&
\sphinxAtStartPar
kgCOD\_S/m3
&
\sphinxAtStartPar
half saturation value of total LCFA
\\
\hline
\sphinxAtStartPar
55
&
\sphinxAtStartPar
K\_Ih2\_fa
&
\sphinxAtStartPar
kgCOD/m3
&
\sphinxAtStartPar
inhibition constant LCFA (substrate) degradation by hydrogen (inhibitor)
\\
\hline
\sphinxAtStartPar
56
&
\sphinxAtStartPar
k\_m\_c4
&
\sphinxAtStartPar
kgCOD\_S/kgCOD\_X/d
&
\sphinxAtStartPar
Monod maximum specific uptake rate for butyrate and valerate
\\
\hline
\sphinxAtStartPar
57
&
\sphinxAtStartPar
K\_S\_c4
&
\sphinxAtStartPar
kgCOD\_S/m3
&
\sphinxAtStartPar
half saturation value of butyrate
\\
\hline
\sphinxAtStartPar
58
&
\sphinxAtStartPar
K\_Ih2\_c4
&
\sphinxAtStartPar
kgCOD/m3
&
\sphinxAtStartPar
inhibition constant for butyrate and valerate (substrate) degradation by hydrogen (inhibitor)
\\
\hline
\sphinxAtStartPar
59
&
\sphinxAtStartPar
k\_m\_pro
&
\sphinxAtStartPar
kgCOD\_S/kgCOD\_X/d
&
\sphinxAtStartPar
Monod maximum specific uptake rate for total propionate
\\
\hline
\sphinxAtStartPar
60
&
\sphinxAtStartPar
K\_S\_pro
&
\sphinxAtStartPar
kgCOD\_S/m3
&
\sphinxAtStartPar
half saturation value of total propionate
\\
\hline
\sphinxAtStartPar
61
&
\sphinxAtStartPar
K\_Ih2\_pro
&
\sphinxAtStartPar
kgCOD/m3
&
\sphinxAtStartPar
inhibition constant for propionate (substrate) degradation by hydrogen (inhibitor)
\\
\hline
\sphinxAtStartPar
62
&
\sphinxAtStartPar
k\_m\_ac
&
\sphinxAtStartPar
kgCOD\_S/kgCOD\_X/d
&
\sphinxAtStartPar
Monod maximum specific uptake rate for total acetate
\\
\hline
\sphinxAtStartPar
63
&
\sphinxAtStartPar
K\_S\_ac
&
\sphinxAtStartPar
kgCOD\_S/m3
&
\sphinxAtStartPar
half saturation value of total acetate
\\
\hline
\sphinxAtStartPar
64
&
\sphinxAtStartPar
K\_I\_nh3
&
\sphinxAtStartPar
kmol N/m3
&
\sphinxAtStartPar
inhibition constant by ammonia (inhibitor)
\\
\hline
\sphinxAtStartPar
65
&
\sphinxAtStartPar
pH\_UL\_ac
&
\sphinxAtStartPar
unitless
&
\sphinxAtStartPar
upper pH limit for total acetate
\\
\hline
\sphinxAtStartPar
66
&
\sphinxAtStartPar
pH\_LL\_ac
&
\sphinxAtStartPar
unitless
&
\sphinxAtStartPar
lower pH limit for total acetate
\\
\hline
\sphinxAtStartPar
67
&
\sphinxAtStartPar
k\_m\_h2
&
\sphinxAtStartPar
kgCOD\_S/kgCOD\_X/d
&
\sphinxAtStartPar
Monod maximum specific uptake rate for hydrogen
\\
\hline
\sphinxAtStartPar
68
&
\sphinxAtStartPar
K\_S\_h2
&
\sphinxAtStartPar
kgCOD\_S/m3
&
\sphinxAtStartPar
half saturation value of hydrogen
\\
\hline
\sphinxAtStartPar
69
&
\sphinxAtStartPar
pH\_UL\_h2
&
\sphinxAtStartPar
unitless
&
\sphinxAtStartPar
upper pH limit for hydrogen
\\
\hline
\sphinxAtStartPar
70
&
\sphinxAtStartPar
pH\_LL\_h2
&
\sphinxAtStartPar
unitless
&
\sphinxAtStartPar
lower pH limit for hydrogen
\\
\hline
\sphinxAtStartPar
71
&
\sphinxAtStartPar
k\_dec\_Xsu
&
\sphinxAtStartPar
1/d
&
\sphinxAtStartPar
first order decay rate for the monosaccharide degraders
\\
\hline
\sphinxAtStartPar
72
&
\sphinxAtStartPar
k\_dec\_Xaa
&
\sphinxAtStartPar
1/d
&
\sphinxAtStartPar
first order decay rate for the amino acids degraders
\\
\hline
\sphinxAtStartPar
73
&
\sphinxAtStartPar
k\_dec\_Xfa
&
\sphinxAtStartPar
1/d
&
\sphinxAtStartPar
first order decay rate for the LCFA degraders
\\
\hline
\sphinxAtStartPar
74
&
\sphinxAtStartPar
k\_dec\_Xc4
&
\sphinxAtStartPar
1/d
&
\sphinxAtStartPar
first order decay rate for the butyrate and valerate
\\
\hline
\sphinxAtStartPar
75
&
\sphinxAtStartPar
k\_dec\_Xpro
&
\sphinxAtStartPar
1/d
&
\sphinxAtStartPar
first order decay rate for the propionate degraders
\\
\hline
\sphinxAtStartPar
76
&
\sphinxAtStartPar
k\_dec\_Xac
&
\sphinxAtStartPar
1/d
&
\sphinxAtStartPar
first order decay rate for the acetate degraders
\\
\hline
\sphinxAtStartPar
77
&
\sphinxAtStartPar
k\_dec\_Xh2
&
\sphinxAtStartPar
1/d
&
\sphinxAtStartPar
first order decay rate for the hydrogen degraders
\\
\hline
\sphinxAtStartPar
78
&
\sphinxAtStartPar
R
&
\sphinxAtStartPar
bar m3 kmole\sphinxhyphen{}1 K\sphinxhyphen{}1
&
\sphinxAtStartPar
gas law constant (8.314e\sphinxhyphen{}2)
\\
\hline
\sphinxAtStartPar
79
&
\sphinxAtStartPar
T\_base
&
\sphinxAtStartPar
°C
&
\sphinxAtStartPar
base temperature
\\
\hline
\sphinxAtStartPar
80
&
\sphinxAtStartPar
T\_op
&
\sphinxAtStartPar
°C
&
\sphinxAtStartPar
operating temperature
\\
\hline
\sphinxAtStartPar
81
&
\sphinxAtStartPar
pK\_w\_base
&
\sphinxAtStartPar
unitless
&
\sphinxAtStartPar
pKa of water
\\
\hline
\sphinxAtStartPar
82
&
\sphinxAtStartPar
pK\_a\_va\_base
&
\sphinxAtStartPar
unitless
&
\sphinxAtStartPar
pKa of total valerate
\\
\hline
\sphinxAtStartPar
83
&
\sphinxAtStartPar
pK\_a\_bu\_base
&
\sphinxAtStartPar
unitless
&
\sphinxAtStartPar
pKa of total butyrate
\\
\hline
\sphinxAtStartPar
84
&
\sphinxAtStartPar
pK\_a\_pro\_base
&
\sphinxAtStartPar
unitless
&
\sphinxAtStartPar
pKa of total propionate
\\
\hline
\sphinxAtStartPar
85
&
\sphinxAtStartPar
pK\_a\_ac\_base
&
\sphinxAtStartPar
unitless
&
\sphinxAtStartPar
pKa of total acetate
\\
\hline
\sphinxAtStartPar
86
&
\sphinxAtStartPar
pK\_a\_co2\_base
&
\sphinxAtStartPar
unitless
&
\sphinxAtStartPar
pKa of carbon dioxide
\\
\hline
\sphinxAtStartPar
87
&
\sphinxAtStartPar
pK\_a\_IN\_base
&
\sphinxAtStartPar
unitless
&
\sphinxAtStartPar
pKa of inorganic nitrogen
\\
\hline
\sphinxAtStartPar
88
&
\sphinxAtStartPar
pK\_a\_hco3\_base
&
\sphinxAtStartPar
unitless
&
\sphinxAtStartPar
pKa of bicarbonate
\\
\hline
\sphinxAtStartPar
89
&
\sphinxAtStartPar
k\_A\_Bbu
&
\sphinxAtStartPar
1/M/d
&
\sphinxAtStartPar
acid base kinetic parameter for total butyrate
\\
\hline
\sphinxAtStartPar
90
&
\sphinxAtStartPar
k\_A\_Bpro
&
\sphinxAtStartPar
1/M/d
&
\sphinxAtStartPar
acid base kinetic parameter for total propionate
\\
\hline
\sphinxAtStartPar
91
&
\sphinxAtStartPar
k\_A\_Bac
&
\sphinxAtStartPar
1/M/d
&
\sphinxAtStartPar
acid base kinetic parameter for total acetate
\\
\hline
\sphinxAtStartPar
92
&
\sphinxAtStartPar
k\_A\_Bco2
&
\sphinxAtStartPar
1/M/d
&
\sphinxAtStartPar
acid base kinetic parameter for carbon dioxide
\\
\hline
\sphinxAtStartPar
93
&
\sphinxAtStartPar
k\_A\_BIN
&
\sphinxAtStartPar
1/M/d
&
\sphinxAtStartPar
acid base kinetic parameter for inhibitors
\\
\hline
\sphinxAtStartPar
94
&
\sphinxAtStartPar
P\_atm
&
\sphinxAtStartPar
bar
&
\sphinxAtStartPar
atmospheric pressure
\\
\hline
\sphinxAtStartPar
95
&
\sphinxAtStartPar
kLa
&
\sphinxAtStartPar
1/d
&
\sphinxAtStartPar
gas\sphinxhyphen{}liquid transfer coefficient
\\
\hline
\sphinxAtStartPar
96
&
\sphinxAtStartPar
K\_H\_h2o\_base
&
\sphinxAtStartPar
M(liq)/bar
&
\sphinxAtStartPar
Henry’s law coefficient of water
\\
\hline
\sphinxAtStartPar
97
&
\sphinxAtStartPar
K\_H\_co2\_base
&
\sphinxAtStartPar
M(liq)/bar
&
\sphinxAtStartPar
Henry’s law coefficient of carbon dioxide
\\
\hline
\sphinxAtStartPar
98
&
\sphinxAtStartPar
K\_H\_ch4\_base
&
\sphinxAtStartPar
M(liq)/bar
&
\sphinxAtStartPar
Henry’s law coefficient of methane
\\
\hline
\sphinxAtStartPar
99
&
\sphinxAtStartPar
K\_H\_h2\_base
&
\sphinxAtStartPar
M(liq)/bar
&
\sphinxAtStartPar
Henry’s law coefficient of hydrogen
\\
\hline
\sphinxAtStartPar
100
&
\sphinxAtStartPar
k\_P
&
\sphinxAtStartPar
m2/d/bar
&
\sphinxAtStartPar
proportional gain
\\
\hline
\end{longtable}\sphinxatlongtableend\end{savenotes}


\section{Outputs {[}indicator***.out{]}}
\label{\detokenize{inouts:outputs-indicator-out}}

\begin{savenotes}\sphinxatlongtablestart\begin{longtable}[c]{|\X{10}{120}|\X{20}{120}|\X{30}{120}|\X{60}{120}|}
\hline
\sphinxstyletheadfamily 
\sphinxAtStartPar
Index
&\sphinxstyletheadfamily 
\sphinxAtStartPar
Notation
&\sphinxstyletheadfamily 
\sphinxAtStartPar
Unit
&\sphinxstyletheadfamily 
\sphinxAtStartPar
Description
\\
\hline
\endfirsthead

\multicolumn{4}{c}%
{\makebox[0pt]{\sphinxtablecontinued{\tablename\ \thetable{} \textendash{} continued from previous page}}}\\
\hline
\sphinxstyletheadfamily 
\sphinxAtStartPar
Index
&\sphinxstyletheadfamily 
\sphinxAtStartPar
Notation
&\sphinxstyletheadfamily 
\sphinxAtStartPar
Unit
&\sphinxstyletheadfamily 
\sphinxAtStartPar
Description
\\
\hline
\endhead

\hline
\multicolumn{4}{r}{\makebox[0pt][r]{\sphinxtablecontinued{continues on next page}}}\\
\endfoot

\endlastfoot

\sphinxAtStartPar
1
&
\sphinxAtStartPar
Ssu
&
\sphinxAtStartPar
mg COD/L
&
\sphinxAtStartPar
soluble  monosaccharides
\\
\hline
\sphinxAtStartPar
2
&
\sphinxAtStartPar
Saa
&
\sphinxAtStartPar
mg COD/L
&
\sphinxAtStartPar
soluble  amino acids
\\
\hline
\sphinxAtStartPar
3
&
\sphinxAtStartPar
Sfa
&
\sphinxAtStartPar
mg COD/L
&
\sphinxAtStartPar
soluble  total LCFA
\\
\hline
\sphinxAtStartPar
4
&
\sphinxAtStartPar
Sva
&
\sphinxAtStartPar
mg COD/L
&
\sphinxAtStartPar
soluble  total valerate
\\
\hline
\sphinxAtStartPar
5
&
\sphinxAtStartPar
Sbu
&
\sphinxAtStartPar
mg COD/L
&
\sphinxAtStartPar
soluble  total butyrate
\\
\hline
\sphinxAtStartPar
6
&
\sphinxAtStartPar
Spro
&
\sphinxAtStartPar
mg COD/L
&
\sphinxAtStartPar
soluble  total propionate
\\
\hline
\sphinxAtStartPar
7
&
\sphinxAtStartPar
Sac
&
\sphinxAtStartPar
mg COD/L
&
\sphinxAtStartPar
soluble  total acetate
\\
\hline
\sphinxAtStartPar
8
&
\sphinxAtStartPar
Sh2
&
\sphinxAtStartPar
mg COD/L
&
\sphinxAtStartPar
soluble  hydrogen
\\
\hline
\sphinxAtStartPar
9
&
\sphinxAtStartPar
Sch4
&
\sphinxAtStartPar
mg COD/L
&
\sphinxAtStartPar
soluble  methane
\\
\hline
\sphinxAtStartPar
10
&
\sphinxAtStartPar
Sic
&
\sphinxAtStartPar
mg C/L
&
\sphinxAtStartPar
soluble  inorganic carbon
\\
\hline
\sphinxAtStartPar
11
&
\sphinxAtStartPar
Sin
&
\sphinxAtStartPar
mg N/L
&
\sphinxAtStartPar
soluble  inorganic nitrogen
\\
\hline
\sphinxAtStartPar
12
&
\sphinxAtStartPar
Si
&
\sphinxAtStartPar
mg COD/L
&
\sphinxAtStartPar
soluble  inerts
\\
\hline
\sphinxAtStartPar
13
&
\sphinxAtStartPar
Xc
&
\sphinxAtStartPar
mg COD/L
&
\sphinxAtStartPar
particulate  composites
\\
\hline
\sphinxAtStartPar
14
&
\sphinxAtStartPar
Xch
&
\sphinxAtStartPar
mg COD/L
&
\sphinxAtStartPar
particulate  carbohydrates
\\
\hline
\sphinxAtStartPar
15
&
\sphinxAtStartPar
Xpr
&
\sphinxAtStartPar
mg COD/L
&
\sphinxAtStartPar
particulate  proteins
\\
\hline
\sphinxAtStartPar
16
&
\sphinxAtStartPar
Xli
&
\sphinxAtStartPar
mg COD/L
&
\sphinxAtStartPar
particulate  lipids
\\
\hline
\sphinxAtStartPar
17
&
\sphinxAtStartPar
Xsu
&
\sphinxAtStartPar
mg COD/L
&
\sphinxAtStartPar
monosaccharides degraders (microorganisms)
\\
\hline
\sphinxAtStartPar
18
&
\sphinxAtStartPar
Xaa
&
\sphinxAtStartPar
mg COD/L
&
\sphinxAtStartPar
amino acids degraders (microorganisms)
\\
\hline
\sphinxAtStartPar
19
&
\sphinxAtStartPar
Xfa
&
\sphinxAtStartPar
mg COD/L
&
\sphinxAtStartPar
LCFA degraders (microorganisms)
\\
\hline
\sphinxAtStartPar
20
&
\sphinxAtStartPar
Xc4
&
\sphinxAtStartPar
mg COD/L
&
\sphinxAtStartPar
butyrate and valerate  degraders (microorganisms)
\\
\hline
\sphinxAtStartPar
21
&
\sphinxAtStartPar
Xpro
&
\sphinxAtStartPar
mg COD/L
&
\sphinxAtStartPar
propionate degraders (microorganisms)
\\
\hline
\sphinxAtStartPar
22
&
\sphinxAtStartPar
Xac
&
\sphinxAtStartPar
mg COD/L
&
\sphinxAtStartPar
acetate degraders (microorganisms)
\\
\hline
\sphinxAtStartPar
23
&
\sphinxAtStartPar
Xh2
&
\sphinxAtStartPar
mg COD/L
&
\sphinxAtStartPar
hydrogen degraders (microorganisms)
\\
\hline
\sphinxAtStartPar
24
&
\sphinxAtStartPar
Xi
&
\sphinxAtStartPar
mg COD/L
&
\sphinxAtStartPar
particulate  inerts
\\
\hline
\sphinxAtStartPar
25
&
\sphinxAtStartPar
scat+
&
\sphinxAtStartPar
mmol/L
&
\sphinxAtStartPar
cations
\\
\hline
\sphinxAtStartPar
26
&
\sphinxAtStartPar
san\sphinxhyphen{}
&
\sphinxAtStartPar
mmol/L
&
\sphinxAtStartPar
anions
\\
\hline
\sphinxAtStartPar
27
&
\sphinxAtStartPar
pH
&
\sphinxAtStartPar
unitless
&
\sphinxAtStartPar
a scale used to specify how acidic or basic a water\sphinxhyphen{}based solution is
\\
\hline
\sphinxAtStartPar
28
&
\sphinxAtStartPar
S\_H+
&
\sphinxAtStartPar
mol/L
&
\sphinxAtStartPar
soluble hydrogen cation
\\
\hline
\sphinxAtStartPar
29
&
\sphinxAtStartPar
Sva\sphinxhyphen{}
&
\sphinxAtStartPar
mg COD/L
&
\sphinxAtStartPar
soluble total valerate anion
\\
\hline
\sphinxAtStartPar
30
&
\sphinxAtStartPar
Sbu\sphinxhyphen{}
&
\sphinxAtStartPar
mg COD/L
&
\sphinxAtStartPar
soluble total butyrate anion
\\
\hline
\sphinxAtStartPar
31
&
\sphinxAtStartPar
Spro\sphinxhyphen{}
&
\sphinxAtStartPar
mg COD/L
&
\sphinxAtStartPar
soluble total propionate anion
\\
\hline
\sphinxAtStartPar
32
&
\sphinxAtStartPar
Sac\sphinxhyphen{}
&
\sphinxAtStartPar
mg COD/L
&
\sphinxAtStartPar
soluble total acetate anion
\\
\hline
\sphinxAtStartPar
33
&
\sphinxAtStartPar
Shco3\sphinxhyphen{}
&
\sphinxAtStartPar
mmol C/L
&
\sphinxAtStartPar
soluble bicarbonate anion
\\
\hline
\sphinxAtStartPar
34
&
\sphinxAtStartPar
Sco2
&
\sphinxAtStartPar
mmol C/L
&
\sphinxAtStartPar
soluble carbon dioxide
\\
\hline
\sphinxAtStartPar
35
&
\sphinxAtStartPar
Snh3
&
\sphinxAtStartPar
mg N/L
&
\sphinxAtStartPar
soluble ammonia
\\
\hline
\sphinxAtStartPar
36
&
\sphinxAtStartPar
Snh4+
&
\sphinxAtStartPar
mg N/L
&
\sphinxAtStartPar
soluble ammonia cation (ammonium)
\\
\hline
\sphinxAtStartPar
37
&
\sphinxAtStartPar
Sgas,h2
&
\sphinxAtStartPar
mg COD/L
&
\sphinxAtStartPar
soluble hydrogen gas
\\
\hline
\sphinxAtStartPar
38
&
\sphinxAtStartPar
Sgas,ch4
&
\sphinxAtStartPar
mg COD/L
&
\sphinxAtStartPar
soluble methane gas
\\
\hline
\sphinxAtStartPar
39
&
\sphinxAtStartPar
Sgas,co2
&
\sphinxAtStartPar
mmol C/L
&
\sphinxAtStartPar
soluble carbon dioxide gas
\\
\hline
\sphinxAtStartPar
40
&
\sphinxAtStartPar
pgas,h2
&
\sphinxAtStartPar
atm
&
\sphinxAtStartPar
partial pressure of gas hydrogen
\\
\hline
\sphinxAtStartPar
41
&
\sphinxAtStartPar
pgas,ch4
&
\sphinxAtStartPar
atm
&
\sphinxAtStartPar
partial pressure of gas methane
\\
\hline
\sphinxAtStartPar
42
&
\sphinxAtStartPar
pgas,co2
&
\sphinxAtStartPar
atm
&
\sphinxAtStartPar
partial pressure of gas carbon dioxide
\\
\hline
\sphinxAtStartPar
43
&
\sphinxAtStartPar
pgas,total
&
\sphinxAtStartPar
atm
&
\sphinxAtStartPar
partial pressure of gas all gases
\\
\hline
\sphinxAtStartPar
44
&
\sphinxAtStartPar
pgas
&
\sphinxAtStartPar
m3/d
&
\sphinxAtStartPar
flow rate of gas
\\
\hline
\sphinxAtStartPar
45
&
\sphinxAtStartPar
Si
&
\sphinxAtStartPar
mg COD/L
&
\sphinxAtStartPar
soluble inert organics
\\
\hline
\sphinxAtStartPar
46
&
\sphinxAtStartPar
Ss
&
\sphinxAtStartPar
mg COD/L
&
\sphinxAtStartPar
readily biodegradable substrate
\\
\hline
\sphinxAtStartPar
47
&
\sphinxAtStartPar
Xi
&
\sphinxAtStartPar
mg COD/L
&
\sphinxAtStartPar
particulate inert organics
\\
\hline
\sphinxAtStartPar
48
&
\sphinxAtStartPar
Xs
&
\sphinxAtStartPar
mg COD/L
&
\sphinxAtStartPar
slowly biodegradable substrate
\\
\hline
\sphinxAtStartPar
49
&
\sphinxAtStartPar
Xd
&
\sphinxAtStartPar
mg COD/L
&
\sphinxAtStartPar
particulate  arising from biomass decay decay
\\
\hline
\sphinxAtStartPar
50
&
\sphinxAtStartPar
Snh
&
\sphinxAtStartPar
mg N/L
&
\sphinxAtStartPar
ammonia and ammonium nitrogen, soluble  the ammonia produced during ammonification process from soluble organic nitrogen
\\
\hline
\sphinxAtStartPar
51
&
\sphinxAtStartPar
Sns
&
\sphinxAtStartPar
mg N/L
&
\sphinxAtStartPar
soluble biodegradable organic nitrogen generated during hydrolysis of particulate biodegradable organic nitrogen, suggesting it is the concentration of soluble organic hydrogen generated during hydrolysis
\\
\hline
\sphinxAtStartPar
52
&
\sphinxAtStartPar
Xns
&
\sphinxAtStartPar
mg N/L
&
\sphinxAtStartPar
particulate biodegradable organic nitrogen generated during hydrolysis of particulate biodegradable organic nitrogen, suggesting it is the concentration of soluble organic hydrogen generated during hydrolysis
\\
\hline
\sphinxAtStartPar
53
&
\sphinxAtStartPar
Salk
&
\sphinxAtStartPar
mg C/L
&
\sphinxAtStartPar
charge balance
\\
\hline
\sphinxAtStartPar
54
&
\sphinxAtStartPar
TSS
&
\sphinxAtStartPar
mg TSS/L
&
\sphinxAtStartPar
total suspended solids
\\
\hline
\sphinxAtStartPar
55
&
\sphinxAtStartPar
VFA\_C2toC5
&
\sphinxAtStartPar
mg COD/L
&
\sphinxAtStartPar
volatile fatty acid from C2 to C5
\\
\hline
\sphinxAtStartPar
56
&
\sphinxAtStartPar
mass\_Sac
&
\sphinxAtStartPar
mg Hac/L
&
\sphinxAtStartPar
acetic acid
\\
\hline
\sphinxAtStartPar
57
&
\sphinxAtStartPar
PAratio
&
\sphinxAtStartPar
kg acetate/ kg acetate equivalent of propionate
&
\sphinxAtStartPar
acetate propionate ratio
\\
\hline
\sphinxAtStartPar
58
&
\sphinxAtStartPar
Alk
&
\sphinxAtStartPar
mg/L CaCO3
&
\sphinxAtStartPar
alkalinity
\\
\hline
\sphinxAtStartPar
59
&
\sphinxAtStartPar
NH3
&
\sphinxAtStartPar
mg N/L
&
\sphinxAtStartPar
ammonia
\\
\hline
\sphinxAtStartPar
60
&
\sphinxAtStartPar
NH4
&
\sphinxAtStartPar
mg N/L
&
\sphinxAtStartPar
ammonium
\\
\hline
\sphinxAtStartPar
61
&
\sphinxAtStartPar
LCFA
&
\sphinxAtStartPar
mgCOD\_LCFA/L
&
\sphinxAtStartPar
long chain fatty acid
\\
\hline
\sphinxAtStartPar
62
&
\sphinxAtStartPar
percentch4
&
\sphinxAtStartPar
\%
&
\sphinxAtStartPar
biogas methane content, methane percentage output, percent by volume
\\
\hline
\sphinxAtStartPar
63
&
\sphinxAtStartPar
energych4
&
\sphinxAtStartPar
\%
&
\sphinxAtStartPar
energy content of CH4 gas, methane energy output, methane converted to COD, percentage of input that’s converted to CH4 energy wise
\\
\hline
\sphinxAtStartPar
64
&
\sphinxAtStartPar
efficiency
&
\sphinxAtStartPar
\%
&
\sphinxAtStartPar
COD removal
\\
\hline
\sphinxAtStartPar
65
&
\sphinxAtStartPar
VFA/ALK
&
\sphinxAtStartPar
g acetate eq./g CaCO3
&
\sphinxAtStartPar
volatile fatty acid to alkalinity ratio
\\
\hline
\sphinxAtStartPar
66
&
\sphinxAtStartPar
ACN
&
\sphinxAtStartPar
kg COD/m3/d
&
\sphinxAtStartPar
acetate capacity number, the ratio between the maximum acetate utilization rate and the average acetate production rate
\\
\hline
\sphinxAtStartPar
67
&
\sphinxAtStartPar
SampleT
&
\sphinxAtStartPar
d
&\\
\hline
\end{longtable}\sphinxatlongtableend\end{savenotes}

\sphinxAtStartPar
\#   examples
\#   join
\subsubsection*{References}
\subsubsection*{Acknowledgements}

\sphinxAtStartPar
The research was supported by the U.S. Department of Energy, Office of Energy Efficiency and Renewable Energy, Bioenergy Technologies Office, under contract DE\sphinxhyphen{}AC02\sphinxhyphen{}06CH11357.



\renewcommand{\indexname}{Index}
\printindex
\end{document}